\documentclass[a4paper,12pt]{article}
\usepackage{graphicx}
\usepackage{enumitem}
\usepackage{titlesec}
\usepackage{geometry}

% Title format
\titleformat{\section}{\large\bfseries}{}{0em}{}

% Page margins
\geometry{left=3cm, right=3cm, top=3cm, bottom=3cm}

\title{Master Thesis Proposal Template with Time Management Strategy}

\author{Johanna Wittig}

\date{\today}

\begin{document}

\maketitle

\section*{Instructions}
This template provides a structured outline for your master thesis proposal and includes a time management exercise. Complete each section with brief responses (1–2 sentences or bullet points) to save time. After completing, push your updates to GitHub as practice.

\section{1. Summary}
What is the main purpose of your research, and why is it significant?\\
The main purpose is to investigate the influence to the high global surface temperatures 2023 of the emitted water vapor and sulfate aerosole by the Hunga Tonga.

\section{2. Research Background}
What previous research provides context for your study, and what gaps does it address?\\ The emitted sulfate aerosols resulted in a negative forcing and leads to a cooling of 0.038°C on the southern hemisphere in 2022 (Schoeberl et al., 2023). However, the water vapor anomaly leads to an increase in the probability of 7\% to exceed 1.5°C by 2026 (Jenkins et al., 2023). 

\section{3. Research Objectives (incl. Research Questions)}
State the main objectives of your research and outline specific research questions.\\
The main objectives are to investigate rather the eruption of the vulcano contributes to the high global surface temperatures 2023 and to see to what extend an eruption would have contributed to the temperature increase in other years (like in years wit stronger ENSO).

\section{4. Methods and Data}
Which methods will you use to answer your research questions, and why are they suitable?\\ I will use model data from the ECHAM6.3–HAM2.3. It is the latest version of the model and is used to do nudged runs.

\section{5. Timeline and Milestones (incl. Gantt Chart)}
What are the main stages of your research, and when do you plan to complete them?\\

\section{Exercise: Time Management Strategy}
Use this section to outline your weekly planning, prioritisation, and reflection. Keep responses short and focused.

\subsection*{Weekly Planning}
\textbf{Prompt:} List 1–2 key tasks you need to complete this week.\\
\textbf{Answer:}
\begin{itemize}
    \item 
    \item 
\end{itemize}

\subsection*{Task Prioritisation}
 Categorise a few tasks as: 1. Urgent and Important, 2. Important but Not Urgent, 3. Urgent but Not Important, 4. Not Urgent and Not Important.\\
\textbf{Answer:}
\begin{itemize}
    \item Urgent and Important: 
    \item Important but Not Urgent: 
    \item Urgent but Not Important: 
    \item Not Urgent and Not Important: 
\end{itemize}

\subsection*{Reflection and Adjustment}
Briefly reflect on your progress this week. What went well? What challenges did you face?\\
\textbf{Answer:} 

\noindent
\textbf{Adjustment for Next Week:} 


\end{document}
